% !TEX TS-program = XeLaTeX

\documentclass[12pt]{article}
\usepackage{polyglossia}
\usepackage{fontenc}
% feature=1 will make some fearute availabel like proof , problem etc avilable 
% That will be useful
\usepackage[banglamainfont=kalpurush.ttf, 
            banglattfont=kalpurush.ttf,
            feature=1
           ]{latexbangla}

% This will make Date, section and subsection and so on to bnagali number numbers  
% if you dont want to convert it to bangali then comment it
\setdefaultlanguage[numerals=Bengali,
changecounternumbering=true]{bengali}
\title{তিতাস ক্যালকুলাস-১। অধ্যায়-১২ চিটশিট }
\author{মোঃ সাকিব হাসান}
\date{\today}


\begin{document}
\maketitle



\newpage

\section{কতিপয় সূত্র যা অবশ্যই জানতে হবে}
\begin{enumerate}[i]
\item $\int\frac{dx}{x^2+a^2}=\frac{1}{a}tan^{-1}(\frac{x}{a})+c$
\item $\int\frac{dx}{a^2-x^2}=\frac{1}{2a}ln|\frac{a+x}{a-x}|+c$
\item $\int\frac{dx}{x^2-a^2}=\frac{1}{2a}ln|\frac{x-a}{x+a}|+c$
\item $\int\frac{dx}{a^2-x^2}=$
\item $\int\frac{dx}{a^2-x^2}=$
\item $\int\frac{dx}{a^2-x^2}=$
\item $\int\frac{dx}{a^2-x^2}=$
\item $\int\frac{dx}{a^2-x^2}=$
\item $\int\frac{dx}{a^2-x^2}=$
\end{enumerate}

\newpage

\section{নিয়ম সমূহঃ}
এই ধরণের ইন্ট্রিগ্রেশনের Equation-এ মূলত দুটি অংশ থাকে। যেমনঃ- ভগ্নাংশ থাকলে লব ও হর নাহয় রুট এর ভেতর ও বাহিরের দুটি পার্ট। আমাদের মূল লক্ষ্য হবে এমন ভাবে সরল করা যাতে Equation-টি উপরের সূত্রগুলোর মতো গঠনে আনতে পারি। এর জন্য বেশিরভাগ সময় আমরা Equation-এর কোন নির্দিষ্ট অংশকে $z$ , $\frac{1}{z}$ , $z^2$ ইত্যাদি ধরে ইন্ট্রিগ্রেশনটি সমাধান করবো

\subsection{দ্বিঘাত সমীকরণের উপস্থিতি}
এই ধরণের গানিতিক সমস্যা এই অধ্যায়ে সবচেয়ে বেশি দেখা যায়। অন্যান্য নিয়মগুলোর যে গানিতিক সমস্যা আছে সেইগুলোর বেশিরভাগের শেষে এই নিয়ম ব্যাবহার করে সমস্যা সমাধান করতে হয়।  \\

\subsubsection*{সমস্যার ধরণঃ} $\int\frac{dx}{\text{দ্বিঘাত সমীকরণ}} $, $\int\frac{dx}{\sqrt{\text{দ্বিঘাত সমীকরণ}}} $, অথবা $\int\sqrt{\text{দ্বিঘাত সমীকরণ}}$ - যেকোনো  এক প্রকারের গঠন থাকলে এই নিয়মটি কার্যকর হবে।
\subsubsection*{সমাধানের নিয়মঃ}
\begin{enumerate}
\item অবশ্যই $x^2$ এর সহগ 1 করিতে হবে। যেমনঃ $5x^2$ থাকলে $x^2$ বানাতে হবে।
\item এর পর দ্বিঘাত সমীকরণকে সরল করে $()^2\pm()^2$ এমন গঠনের আনতে হবে।
\item এর পর উপরোক্ত সূত্রগুলো প্রয়োগ করলেই সমাধান হয়ে যাবে
\end{enumerate}
\subsubsection*{ দ্বিঘাত সমীকরণকে সরল করার জন্য প্রয়োজনীয় সুত্রঃ} 
\begin{enumerate}
\item $(a+b)^2=a^2+2ab+b^2$
\item $(a-b)^2=a^2-2ab+b^2$
\item $a^2+b^2=(a+b)^2-2ab =(a-b)^2+2ab$
\item $a^2-b^2=(a+b)(a-b)$
\item $(a-b)^2=(a+b)^2-4ab$
\item $(a+b)^2=(a-b)^2+4ab$
\end{enumerate}
এই সকল সূত্র প্রয়োগের পর সমীকরণটি সমতা করার জন্য বেশির ভাগ সময় কোণ কিছু যোগ কিংবা বিয়োগ করতে হবে। আবার কোন পূর্ণ সংখ্যাকে বর্গমূল আকারে লেখতে হবে। যেমনঃ $\frac{1}{3}$ কে  $(\frac{1}{\sqrt{3}})^2$ বানিয়ে লিখতে হবে। যাতে সমীকরণটির দ্বিঘাত অংশটি $()^2\pm()^2$ আকার ধারণ করে ।
\end{document}